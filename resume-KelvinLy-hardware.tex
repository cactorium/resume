\documentclass{my_resume}
\usepackage{hyperref}
\usepackage[8pt]{extsizes}
\usepackage[letterpaper, margin=0.25in]{geometry}
\thispagestyle{empty}
\input{contactinfo}

\begin{document}

\contact{Kelvin Ly}{kelvin.ly1618@gmail.com}{\myphonenumber}

\education{University of Central Florida}{2016-}
    {PhD, Computer Engineering}{N/A}
\education{University of Central Florida}{2011-2015}
	{BSEE, Electrical Engineering}{3.905, Magna Cum Laude}

\section{Professional Experience}
\datedsubsection{University of Central Florida Undergraduate/Graduate Research, Orlando Fl}
    {November 2015-}
\begin{flushleft}
The focus on my research here has been on the security of the \textbf{Internet of Things}, more specifically the development of defenses for IoT devices against attacks.
Consequently, much of my work so far has been in \textbf{PCB design and assembly} to develop devices to test out security ideas or provide education on hardware security. 
Designs so far have incorporated \textbf{MSP430} and \textbf{Atmel} microcontrollers, and work is on a new design incorporating the \textbf{CC3200} Wifi SoC.
\end{flushleft}
\datedsubsection{University of Central Florida Undergraduate Researcher, Orlando FL}
	{December 2014 - March 2015}
\begin{flushleft}
This research experience actually has had so far two major projects. The first
phase was focused on Working on a \textbf{RAVEN II} medical robot running the
\textbf{ROS C++} robotics framework. This robot was meant to work as surgery
robot, with our task being to augment the controls with BCI-based controls to
improve usability. Unfortunately the robot proved hard to use, and we switched
over to working in \textbf{signal processing} in \textbf{Python} of EEG data in
general. Our team has been studying \textbf{feature extraction} and
\textbf{SSVEP frequency detection} to hopefully advance the state of the art.
We have used \textbf{emokit} \textbf{Python} library to extract signals from
Emotiv EEG headset.
\end{flushleft}
\section{Internships}
\datedsubsection{IBM Extreme Blue Intern, RTP NC}{May 2015 - August 2015}
\begin{flushleft}
Here our team worked on zero knowledge \textbf{encryption} for 
\textbf{IBM Connections Cloud}. We were the pioneering efforts at this, producing
a proof of concept to pave the way for the actual Connections team to develop.
We used \textbf{JavaScript and Node.js} for the server \textbf{backend}, and
modified and used existing \textbf{Java} and \textbf{Python} code and libraries
for various parts of the project. Our team was organized around modern programming
practices, working in an \textbf{agile} team of four, with heavy emphasis on
\textbf{test coverage} and \textbf{unit testing}.
\end{flushleft}
\datedsubsection{Google Software Engineer Intern, Chapel Hill NC}{May 2014 - August 2014}
\begin{flushleft}
Here I worked as an intern on the Skia benchmarking team, worked on benchmarking
framework for \textbf{Skia} rendering engine team. I learned \textbf{Go}, and
contributed code in \textbf{C++}, \textbf{Python}, and \textbf{Go} for both
internal and open source projects. This job involved pipelining the gigabytes
of data being produced daily from test bots into a useful visualization for the
Skia team.
\end{flushleft}
\section{Projects}
\begin{itemize}
    \item \textbf{UCF Lunar Knights} project, electrical/communications teams
        \begin{itemize}
            \item Helped with \textbf{wireless communication} with
                \textbf{Beaglebone Black}
            \item \textbf{UART} communication with \textbf{Arduino} to send
                \textbf{PWM} to motor controllers
            \item Helped in robot assembly, troubleshooting and debugging
        \end{itemize}
    \item \textbf{IEEE-UCF Hardware Team} for SouthEastCon, motors team
        \begin{itemize}
            \item Involved in the design and construction of motors system for
                competition robot
            \item Programmed, along with a few others, the \textbf{Arduino}
                powering the robot during competition
        \end{itemize}
    \item Senior design project
        \begin{itemize}
            \item \textbf{Hardware system design} for all components
            \begin{itemize}
                \item Led overall hardware system design
                \item Designed schematics for all components using \textbf{KiCAD} EDA software
                \item Converted schematics into PCBs using \textbf{KiCAD}
            \end{itemize}
            \item Research into \textbf{signal processing} for \textbf{feature extraction}
                with respect to applications in \textbf{brain-computer interfaces}
            \item Created and designed laser cut design to create gimbal to control wheelchair joystick
            \item Wrote \textbf{assembly} for the \textbf{MSP430} to test the gimbal
        \end{itemize}
\section{Skills}
    \item Hobbyist experience with eletronics design and reverse engineering, guitar electronics
        repair
    \item Fluent in \textbf{C/C++}, \textbf{Python}, \textbf{Go}, \textbf{Verilog}
	\item Working knowledge of \textbf{x86}/\textbf{x64}/\textbf{MIPS}/\textbf{MSP430} assembly,
            \textbf{Java}, \textbf{LaTeX}, \textbf{bash}, \textbf{MATLAB}
            \textbf{Kicad EDA} Software Suite, \textbf{Multisim}, \textbf{Xilinx ISE}
    \item GitHub user: \url{https://github.com/cactorium}
\end{itemize}
\end{document}
